\documentstyle[12pt,proof]{article}

\setlength{\oddsidemargin}{0mm}
%\setlength{\evensidemargin}{0mm}
\setlength{\evensidemargin}{-2mm}
\setlength{\topmargin}{-16mm}
\setlength{\textheight}{240mm}
\setlength{\textwidth}{158mm}

\newcommand{\bequ}{\begin{quote}}
\newcommand{\enqu}{\end{quote}}
\newcommand{\bece}{\begin{center}}
\newcommand{\ence}{\end{center}}
\newcommand{\sugmap}[1]{\mbox{$#1^{\mbox{\scriptsize o}}$}}

\newcommand{\PESC}{PESCA}

\title{{\bf \PESC\ --- A Proof Editor for Sequent Calculus}}

\date{Second Version, March 24, 2000}

\author{Aarne Ranta\\
        {\normalsize \verb6Aarne.Ranta@cs.chalmers.se6}}

\begin{document}

\maketitle

\small

{\bf Abstract}. \PESC\ is a program that helps in the construction of proofs
in sequent calculus. It works both as a proof editor and as an automatic theorem 
prover. Proofs constructed in \PESC\ can both be seen on the terminal and printed
into \LaTeX\ files.
The user of \PESC\ can choose among different versions of classical and 
intuitionistic proposition and predicate calculi, and extend them by systems of 
nonlogical axioms. The aim of this document is to show how to use \PESC. It will
also give an outline of the implementation of \PESC, which is written in the 
functional programming language Haskell.


\normalsize


\section{Introduction}

Sequent calculus was introduced by Gentzen (1934) as a formalization of logic
suitable for metamathematical considerations, such as consistency 
proofs. Gentzen also realized that sequent calculus gives a method of automatic
proof search for intuitionistic proposition calculus. Both of these aspects
were built upon in later works on proof theory, such as in Kleene (1952) and
Sch\"utte (1977), and sequent calculus is still a living subject. At the same time,
it was already realized by Gentzen that sequent calculus is not very
natural for humans actually to write proofs. It carries around a lot
of information that humans tend to keep in their heads rather than to put on 
paper. While greatly improving the performance of machines operating on proofs,
this information easily obscures the human inspection of them, and actually writing
sequent calculus proofs in full detail is tedious and error-prone. Thus it
is obviously a task where a machine can be helpful.

This document will not give an introduction to sequent calculus, let alone to
logic. In developing the system \PESC, we have followed the book manuscript
Negri \&\ von Plato (1999), but the
sufficient background can also be gained from Troelstra \&\ Schwichtenberg
(1996). The system \PESC\ is not, however, fully compatible with the 
``older generation'' of sequent calculi {\em \`a la}\ Gentzen and Sch\"utte,
for reasons that will become clear in the following.

The domain of sequent calculi allows for indefinitely many variations, which are
not due to disagreements on what should be provable but to different decisions on
the fine structure of proofs. In terms of provability, it is usually enough to tell
whether a calculus is {\bf intuitionistic}\ or {\bf classical}. In the properties of
proof structure, there are many more choices. The very implementation of \PESC\
precludes most of them, but it still leaves room for different calculi, only some of
which are included in the basic distribution. These calculi can be characterized as
having
\bece
shared multiset contexts, no structural rules.
\ence
But calculi can have single as well as multiple formulae in the succedents
of their sequents. 

The fundamental common property of the calculi treated by 
\PESC\ is {\bf top-down determinacy}:
\bece
given a conclusion and a rule, the premises are determined.
\ence
This property is essential for our method of {\bf top-down proof search}. The user
of \PESC\ starts with a conclusion, and then tries to {\bf refine}\ it by suggesting
a rule. If the rule is applicable, the proof of the conclusion is completed by
the derived premises, and the proof search can continue from them. A branch in a
proof is successfully terminated when a rule gives an empty list of premises.
A branch fails if no rule applies to it. This simple procedure, which we have 
adopted from the proof editor ALF (Magnusson 1994) for the much richer formalism of 
constructive type theory, is not applicable to calculi that are not top-down 
determinate.

(The term top-down runs counter to the standard typographical layout of proof 
trees, where premises appear above conclusions.
The term is frequent in computer science, where it may come from the
standard layout of syntax trees, where the root is above the trees.
In proof theory, the confusion is usually avoided by saying ``root first'' instead,
as do Negri \&\ von Plato (1999).)


\section{Two example sessions}

\subsection{A proof in proposition calculus}

Let us first construct a proof of the law of commutativity for disjunction,
in the sequent form,
\[
A \vee B \Rightarrow\ B \vee A.
\]
Having started \PESC\ (see Section 5 below to find out how), we see its prompt
\verb6|-6. We then enter a 
{\bf new goal}\ by the command {\tt n}\ followed by the sequent written in
ASCII notation:
\begin{verbatim}
  |- n A v B => B v A
\end{verbatim}
The reply of \PESC\ is a new {\bf proof tree}, consisting of the conclusion alone, 
which is the only {\bf subgoal}\ of the tree. Subgoals are identified by sequences
of digits starting from the root, as in the following example:
\[
\infer{1}{
 \infer{11}{111 & 112}
 &
 \infer{12}{121}
 &
 13
 }
\]
The command {\tt s}\ shows the current subgoals, of which we still have just one,
numbered $1$. More interestingly, the command {\tt a}\ shows the {\bf applicable 
rules}\ for a given subgoal. In the situation where we are in our proof, we have
\begin{verbatim}
  |- a 1

  r 1 A1 S1 Lv  -- A v B => B v A
  r 1 A1 S1 Rv1 -- A v B => B v A
  r 1 A1 S1 Rv2 -- A v B => B v A
\end{verbatim}
Any of the displayed {\tt r}\ commands (line segments preceding \verb6--6) 
can be cut and pasted to a command line, and it gives rise to a {\bf refinement}\
of the subgoal, an extension of the tree that is determined by the chosen rule.
For instance, choosing the first alternative takes us to a proof by the left
disjunction rule from two premises,
\begin{verbatim}
  |- r 1 A1 S1 Lv

  A => B v A   B => B v A
  -----------------------
  A v B => B v A
\end{verbatim}
As will be explained below, 
the part {\tt A1 S1}\ specifies the active formulae in the
antecedent and the succedent; when the first formulae are chosen, like here, this
part could be omitted, like in the next refinement,
\begin{verbatim}
  |- r 11 Rv2
  
  A => A
  ----------
  A => B v A   B => B v A
  -----------------------
  A v B => B v A
\end{verbatim}
The left branch {\tt 111}\ can now be refined by the {\tt ax}\ rule, 
which does not generate any more subgoals:
\begin{verbatim}
  |- r 111 ax
\end{verbatim}
The right branch {\tt 12}\ can be refined analogously with {\tt 11}, by
{\tt Rv1}\ followed by {\tt ax}. When the proof is ready, its ASCII appearance
is usually not of very good quality. Now, at the latest, it is rewarding to use
the {\tt l}\ command to write the current proof into a \LaTeX\ document, which
looks as follows when processed:
\[
\infer[{\scriptstyle L\vee }]{
A \vee  B \Rightarrow  B \vee  A
}{
\infer[{\scriptstyle R\vee 2}]{
A \Rightarrow  B \vee  A
}{
A \Rightarrow  A^{ax}
}
 & 
\infer[{\scriptstyle R\vee 1}]{
B \Rightarrow  B \vee  A
}{
B \Rightarrow  B^{ax}
}
}
\]


\subsection{A proof in predicate calculus}

In predicate calculus, one more command is usually needed than in propositional
calculus: the command {\tt i}\ for {\bf instantiating}\ parametres. Logically, a 
parametre such as {\tt t}\ in the existential introduction rule $R\exists$
\[
\infer{\Gamma \Rightarrow (\exists x)A}{\Gamma \Rightarrow A(t/x)}
\]
is just like another premise, which calls for a construction to be complete.
This logic is made explicit in the $\Sigma$ introduction rule of Martin-L\"of's 
type theory, and simplifies greatly the implementation of inference rules.
Here, staying faithful to the syntax of predicate calculus, we have to treat
such parametres as hidden premises in the rules. Thus a proof that has 
uninstantiated parametres is incomplete in the same way as a proof that has 
open subgoals. 

Let us prove the quantifier switch law,
\[
(\exists y)(\forall x)C(x,y) \Rightarrow (\forall x)(\exists y)C(x,y).
\]
After introducing the goal, we make a couple of ordinary refinements:
\begin{verbatim}
  |- n (/Ey)(/Ax)C(x,y) => (/Ax)(/Ey)C(x,y)
  |- r 1 A1 S1 R/A
  |- r 11 L/E
  |- r 111 A1 S1 R/E
\end{verbatim}
The last refinement introduces a parametre {\tt t}, which we instantiate by 
{\tt y},
\begin{verbatim}
  |- i t y
\end{verbatim}
Continuing by {\tt L/A}, {\tt x}, and {\tt ax}, we obtain a complete proof,
in which we have afterwards marked the two instantiations made:
\[
\infer[{\scriptstyle R\forall }]{
(\exists y)(\forall x)C(x,y) \Rightarrow  (\forall x)(\exists y)C(x,y)
}{
\infer[{\scriptstyle L\exists }]{
(\exists y)(\forall x)C(x,y) \Rightarrow  (\exists y)C(x,y)
}{
\infer[{\scriptstyle R\exists y}]{
(\forall x)C(x,y) \Rightarrow  (\exists y)C(x,y)
}{
\infer[{\scriptstyle L\forall x}]{
(\forall x)C(x,y) \Rightarrow  C(x,y)
}{
C(x,y), (\forall x)C(x,y) \Rightarrow  C(x,y)^{ax}
}
}
}
}
\]


\section{The command language}

This section gives a systematic description of all \PESC\ commands. Some of them
were already exemplified in the two sessions of the previous section.

Each command consist of one character followed by zero or more arguments, 
some of which may be optional. In the following, as in the on-line help
file of \PESC, the arguments are denoted by words indicating their
types. Optional arguments are enclosed in brackets. Typewriter font is used for 
terminal symbols.

Fully to understand the commands, one should know that a \PESC\ session takes place
in an {\bf environment}, which changes as a function of the commands. The 
environment consists of a {\bf current calculus}\ and a {\bf current proof}.
In the beginning, the calculus is the intuitionistic predicate calculus 
G3i (cf.\ Appendix), and the proof is the one consisting of the impossible 
empty sequent $\Rightarrow$. While many of the commands affect the current proof, 
only two of them affect the current calculus.

\newcommand{\ssubsection}[2]{\bequ #2 (#1) \enqu}

\ssubsection{refine}{\verb6r6 goal [\verb6A6 int] [\verb6S6 int] rule}
replaces the goal by an application of the rule, if applicable,
and leaves the current proof unchanged otherwise. The goal is denoted by a 
sequence of digits, as explained in the previous section.
The options \verb6[A6 int\verb6]6 \verb6[S6 int\verb6]6 reset the active formulae
in the antecedent and the succedent of the goal---by default, the active formula is
number 1, which in the antecedent is the first and in the succedent the last 
printed formula. If the number exceeds the length, resetting the active formula has 
no effect.

\ssubsection{instantiate}{\verb6i6 parametre term}
replaces all occurrences of the parametre in the current 
proof by the term. The term can be any variable or constant or other parametre, 
or a complex term built by function applications. Notice that, if a parametre
occurs in the term, it must be prefixed by the character {\tt ?}, so that it
is not parsed as a variable (cf.\ Section 4.3 below).

\ssubsection{try to refine}{\verb6t6 goal int}
replaces the goal by the first proof that it finds by recursively
trying to apply all rules maximally int times. This is the 
{\bf automatic proof search method}\ of \PESC, based on brute force but 
always terminating. With
certain calculi, such as G4ip, this method always finds a proof
after a predictable number of steps. With predicate calculus rules that require 
instantiations, the method usually fails.

\ssubsection{new}{\verb6n6 sequent}
replaces the current proof by a proof consisting of the sequent that is
its open subgoal number 1. The old proof cannot be restored. But the work is
not completely lost: one can save the history of the session by the command 
{\tt h}\ and then follow the same steps to construct again whatever was
constructed in the same session.

\ssubsection{undo}{\verb6u6 subtree}
replaces the subtree by a goal consisting of the conclusion of the subtree.
Subtrees are identified by sequences of digits in the same way as subgoals.

\ssubsection{show subgoals}{\verb6s6}
shows all open subgoals in the format `goal {\tt = }\ sequent'. If the system
responds by showing nothing, the current proof is complete.

\ssubsection{applicable rules}{\verb6a6 goal}
shows all refinement commands applicable to the goal. To each command, the
corresponding permutation of the goal sequent is appended, separated by \verb6--6.

\ssubsection{change calculus}{\verb6c6 calculus}
changes the current calculus. The help command {\tt ?}\ shows available calculi,
and a set of them is also shown in the Appendix. As a calculus is just a set 
of rules, calculi can be unioned by the operation {\tt +}. Thus the command
\verb6c G3c + Geq6 selects classical predicate calculus with equality.

\ssubsection{read axioms}{\verb6x6 file}
reads a file with nonlogical axioms, parses it into rules, adds the rules into
the current calculus, and writes the rules into the file {\tt myaxioms.tex}.
The command also makes a system call to perform the \LaTeX\  processing and show
the xdvi imade on the background.
The theory of nonlogical axioms is explained in chapter 6 of Negri \&\ von Plato 
(1999). The syntax of axiom files is explained in Section 4.4 below.

\ssubsection{print proof in a \LaTeX\ file}{\verb6l6 [file]}
prints the current proof in \LaTeX\ format in the indicated file. If no file
name is given, the name is {\tt myproof.tex}. Old files are overwritten without 
warning. To process the \LaTeX\ file, the style file {\tt proof.sty}\ 
(Tatsuta 1990) is needed. It is distributed together with \PESC.
The command also makes a system call to perform the \LaTeX\  processing and show
the xdvi imade on the background.

\ssubsection{print proof in natural deduction}{\verb6d6 [file]}
prints the current proof in natural deduction 
\LaTeX\ format in the indicated file. If no file
name is given, the name is {\tt mydeduction.tex}. Old files are overwritten without 
warning. To process the \LaTeX\ file, the style file {\tt proof.sty}\ 
(Tatsuta 1990) is needed. It is distributed together with \PESC.
The command also makes a system call to perform the \LaTeX\  processing and show
the xdvi imade on the background.
The command only works for the calculi G3i and G3ip.

\ssubsection{print history in a file}{\verb6h6 [file]}
prints the sequence of command lines entered during the session into file.
The {\tt h}\ command line itself is not included.
The default file name is {\tt myhistory.txt}. Old files are overwritten without 
warning.

\ssubsection{help}{\verb6?6}
shows a synopsis of available commands.

\ssubsection{manual}{\verb6m6}
runs \LaTeX\ on the manual file {\tt manual.tex} and shows the xdvi image on the
background.

shows a synopsis of available commands.

\ssubsection{quit}{\verb6q6}
terminates the \PESC\ session. 
The sequence of command lines entered during the session (except the 
{\tt q}\ itself) are written into the file {\tt myhistory.txt}.
Old files are overwritten without warning.



\newcommand{\kleene}[2]{$\{$ #1 $\}^{\mbox{\tt #2}}$}
\newcommand{\gives}{$\; ::= \; $}
\newcommand{\altern}{$\mid$}
\newcommand{\option}[1]{\sugmap{\mbox{#1}}}
\newcommand{\eg}{}%{e.g.\ }



\section{The syntax of sequents, formulae, and axioms}

The only occasion in which the user of \PESC\ has to type in a sequent is as
an argument of a refine ({\tt r}) command. In addition, terms must be typed
in instantiation ({\tt i}) commands. Proofs and inference rules need never be typed.
If the user wants to enter his own nonlogical axioms from an axiom file,
he will have to follow a limited syntax of formulae. This section will explain the
syntax of sequents, formulae, and terms that is recognized by the \PESC\ parser.
The expressions that \PESC\ prints out may differ from this syntax; the \LaTeX\
output certainly does.

{\bf Junk}, that is, spaces, tabulators, and newlines, may appear between
any items, but not, of course, inside identifiers. In some cases, the presence of
junk is necessary for the parser to tell where one identifier ends and the next 
one begins.

The syntax is given in Backus-Naur style, using \altern\ for alternatives,
\option{}\ for optionality, and \kleene{}{t}\ for lists separated by the token
{\tt t}. The last column of each table shows
an example string satisfying the definition of that row.



\subsection{Sequents and formulae}

A sequent is a pair of (possibly empty) contexts separated by the double 
arrow \verb6=>6. Contexts are lists of formulae separated by commas. 
\bequ
\begin{tabular}{lcl|l}
Sequent & \gives\  &  \option{Context}\ \verb6=>6 \option{Context} 
                   &  \eg \verb6A, B => B, C, A6 \\
Context & \gives\  &  \kleene{Formula}{,} &  \eg \verb6A & B, C6
\end{tabular}
\enqu
Formulae themselves are those of predicate calculus: conjunctions, disjunctions, 
implications, negations, and universal and existential quantifications.
Logically simple formulae fall into three categories: predications,
schematic formulae, and the absurdity $\bot$, of which the last-mentioned one is not
counted as an atomic formula.
\bequ
\begin{tabular}{lcl|l}
Formula & \gives\  &  Atom & \eg \verb6Apt(a,b)6 \\
        & \altern\ &  Formula \verb6&6  Formula & \eg \verb6A & B6  \\
        & \altern\ &  Formula \verb6v6  Formula & \eg \verb6A v B6  \\
        & \altern\ &  Formula \verb6->6 Formula & \eg \verb6A -> B6  \\
        & \altern\ &  \verb6~6 Formula & \eg \verb6~A6  \\
        & \altern\ &  \verb6_|_6 & \eg \verb6_|_6 \\
        & \altern\ &  \verb6(/A6 Var \verb6)6 Formula & \eg \verb6(/Ax)B(x)6  \\
        & \altern\ &  \verb6(/E6 Var \verb6)6 Formula & \eg \verb6(/Ex)B(x)6  \\
Atom    & \gives\  &  Scheme \option{ArgList} & \eg \verb6A(x,y,z)6  \\
        & \altern\ &  Predicate \option{ArgList} & \eg \verb6Apt(x,y)6  \\
        & \altern\ &  Term InfixPred Term & \eg \verb62 < 36  
\end{tabular}
\enqu
Formulae are grouped in accordance with their {\bf precedences},
which are the usual ones, from stronger to weaker: 
simple and quantified and negation, conjunction and disjunction, implication.
These precedences are overridden by the use of parentheses.


\subsection{Argument lists and terms}

Argument lists are lists of terms in parentheses, and terms are either
parametres or variables or constants, the last-mentioned oprionally followed by
an argument list.
\bequ
\begin{tabular}{lcl|l}
ArgList & \gives\  &  \verb6(6 \kleene{Term}{,}  \verb6)6  & \eg \verb6(x,y,z)6 \\
Term    & \gives\  &  Variable  & \eg \verb6x6 \\
        & \altern\ &  Parametre  & \eg \verb6?c6 \\ 
        & \altern\ &  Constant \option{ArgumentList}  & \eg \verb6pi6 \\
        & \altern\ &  Term InfixFun Term  & \eg \verb6x + 26 
\end{tabular}
\enqu
A notion of precedence would make sense for infix terms, but there is so far none
in \PESC. Nested infix terms should thus be disambiguated by parentheses.


\subsection{Identifiers}

In the syntactic rules above, several classes of identifiers have been presupposed.
The parser of \PESC\ keeps them disjoint, to avoid the need of lookup in a 
dictionary. Thus they are different combinations of capital and small letters,
digits, and other characters. No junk (e.g.\ space) is tolerated inside an 
identifier.
\bequ
\begin{tabular}{lcl|l}
Scheme    & \gives\  & \kleene{Capital}{}  & \eg \verb6A6 \\
Predicate & \gives\  & Capital \kleene{Small}{}  & \eg \verb6Apt6 \\
Variable  & \gives\  & Small \kleene{\verb6'6}{}  & \eg \verb6x''6 \\
Parametre & \gives\  & \verb6?6 Variable  & \eg \verb6?c''6 \\
Constant  & \gives\  & Small \kleene{Small}{}  & \eg \verb6pi6 \\
          & \altern\ & \kleene{Digit}{}  & \eg \verb61236 \\
InfixPred & \gives\  & \verb6=6 \altern\ \verb6<6 \altern\ \verb6>6 
                                \altern\ \verb6#6  & \eg \verb6>6 \\
          & \altern\ & \verb6\6 \kleene{Letter}{}  & \eg \verb6\leq6 \\
InfixFun  & \gives\  & \verb6+6 \altern\ \verb6-6 \altern\ \verb6*6 
                     & \eg \verb6+6 \\
          & \altern\ & \verb6\6 \kleene{Letter}{}  & \eg \verb6\circ6 \\
\end{tabular}
\enqu
Notice that the definition of infix predicates and functions allows the
use of the mathematical symbolism of \LaTeX. Also notice that, since
there is no distinction in \LaTeX\ symbols between predicates and 
individual functions, parentheses are needed if both are present.
For instance, the \LaTeX/\PESC\ formula
\begin{verbatim}
  c \leq a \wedge b
\end{verbatim}
is ambiguous between whether \verb6\wedge6 or \verb6\leq6 is the predicate,
and the normal way of reading it is achieved by
\begin{verbatim}
  c \leq (a \wedge b)
\end{verbatim}
Since the intended structure is clear for the human reader, the \LaTeX\ output is 
still
\[
c \leq a \wedge b
\]



\subsection{Axioms}

As shown in section 6 of Negri \&\ von Plato (1999), certain kinds of formulae
can be interpreted as sequent calculus rules that preserve the possibility of
cut elimination and are hence favourable for proof search. These formulae are
implications with conjunctions of atoms on their left-hand sides, and
disjunctions of atoms on their right-hand sides. Either side can be empty.
An empty left-hand side is represented by the omission of the implication sign.
An empty right-hand side is represented by the absurdity $\bot$. 
Alternatively, the negation of a left-hand side is interpreted as its
impication of absurdity; parentheses are not strictly necessary, 
since the whole conjunction must be in the scope of the negation.
\bequ
\begin{tabular}{lcl|l}
Axiom     & \gives\  &  LeftSide  \verb6->6  RightSide & \eg \verb6x=y -> y=x6 \\
          & \altern\ &  RightSide & \eg \verb60=06 \\
          & \altern\ &  \verb6~6 LeftSide & \eg \verb6~ x#x6 \\
LeftSide  & \gives\  &  \kleene{Atom}{\&}  & \eg \verb6x<y & y<z6 \\
RightSide & \gives\  &  \kleene{Atom}{v}   & \eg \verb6x#z v y#z6 \\
          & \altern\ &  \verb6_|_6         & \eg \verb6_|_6 
\end{tabular}
\enqu
Atoms are like in Section 4.1 above, with two exceptions:
\bequ
Schematic letters are not permitted in atoms, but only constant predicates.

Parametres (variables prefixed by \verb6?6) are not recognized in atoms. 
Instead, all those variables that occur on the right-hand-side but not 
on the left-hand-side are treated as parametres.
\enqu
In axiom files, each axiom is preceded by an identifier to be used as
a rule label. This identifier can contain any characters except junk.
The second example of the following section shows an example axiom file.

An axiom file may contain {\em comments}, which are lines beginning with
\verb6--6. Comments are ignored, except the first one, which has a special effect:
the part of the file preceding that comment is passed to \LaTeX\ as it is,
permitting the user to define macros for notation.
Thus an axiom file must contain at least one comment;
otherwise it is interpreted as empty.


\subsection{Examples of sequents and axioms}

Here are some well-formed \PESC\ sequents and their \LaTeX\ printouts:
\bequ
\verb6=>6

$\Rightarrow$

\verb6A & B => A, B6 

$A \&  B \Rightarrow  A, B$

\verb6=> ~~(A & B) -> ~~A & ~~B6

$\Rightarrow \sim\sim (A \& B) \supset \sim \sim A \& \sim \sim B$

\verb6(/Ey)(/Ax)C(x,y) => (/Ax)(/Ey)C(x,y)6

$(\exists y)(\forall x)C(x,y) \Rightarrow  (\forall x)(\exists y)C(x,y)$

\verb6=> (/Ax)(x > 0 -> (/Ey)(y \leq (2 \div x) & y \geq 0))6

$\Rightarrow (\forall x)(x > 0 \supset (\exists y)(y \leq 2 \div x \& y \geq 0))$
\enqu
The second example is an axiom file and the corresponding rules in \LaTeX\ output:
some axioms of the theory of lattices from section 6.6 of 
Negri \&\ von Plato (1999):
\begin{verbatim}
  %% this time no macros for lattice theory
  -- axioms of lattice theory
  Mtl    (a \wedge b) \leq a
  Mtr    (a \wedge b) \leq b
  Jnl    a \leq (a \vee b)
  Jnr    b \leq (a \vee b)
  Unimt  c \leq a & c \leq b -> c \leq (a \wedge b)
  Unijn  a \leq c & b \leq c -> (a \vee b) \leq c 
  Ref    a \leq a 
  Trans  a \leq b & b \leq c -> a \leq c
\end{verbatim}
\[
\infer[{\scriptstyle Mtl}]{
\Gamma  \Rightarrow  \Delta 
}{
a \wedge b \leq a, \Gamma  \Rightarrow  \Delta 
}
\]
\[
\infer[{\scriptstyle Mtr}]{
\Gamma  \Rightarrow  \Delta 
}{
a \wedge b \leq b, \Gamma  \Rightarrow  \Delta 
}
\]
\[
\infer[{\scriptstyle Jnl}]{
\Gamma  \Rightarrow  \Delta 
}{
a \leq a \vee b, \Gamma  \Rightarrow  \Delta 
}
\]
\[
\infer[{\scriptstyle Jnr}]{
\Gamma  \Rightarrow  \Delta 
}{
b \leq a \vee b, \Gamma  \Rightarrow  \Delta 
}
\]
\[
\infer[{\scriptstyle Unimt}]{
c \leq a, c \leq b, \Gamma  \Rightarrow  \Delta 
}{
c \leq a \wedge b, c \leq a, c \leq b, \Gamma  \Rightarrow  \Delta 
}
\]
\[
\infer[{\scriptstyle Unijn}]{
a \leq c, b \leq c, \Gamma  \Rightarrow  \Delta 
}{
a \vee b \leq c, a \leq c, b \leq c, \Gamma  \Rightarrow  \Delta 
}
\]
\[
\infer[{\scriptstyle Ref}]{
\Gamma  \Rightarrow  \Delta 
}{
a \leq a, \Gamma  \Rightarrow  \Delta 
}
\]
\[
\infer[{\scriptstyle Trans}]{
a \leq b, b \leq c, \Gamma  \Rightarrow  \Delta 
}{
a \leq c, a \leq b, b \leq c, \Gamma  \Rightarrow  \Delta 
}
\]


\section{Usage and compiling}

\subsection{Running \PESC}

The executable \PESC\ program consists of a shell script, named
{\tt pesca}.
If this file exists in one's path, one can start a \PESC\ session by
just typing the name of the file and pressing enter in the shell:
\begin{verbatim}
  shell$ ./pesca
\end{verbatim}
The script calls the Haskell
interpreter {\tt hugs}\ in the batch mode, {\tt runhugs}.
To obtain hugs, consult the hugs home page mentioned in the references.



\subsection{Compiling \PESC}

\PESC\ is a small and light program, which hardly needs to be compiled to
be usable. Anyone who wishes to do so, however, can use
e.g.\ {\tt hbc}\ or {\tt ghc}. The {\tt Main}\ module lies in the file
{\tt Editor.hs}.



\section{The implementation}

This section will go through the basic data structures and algorithms of \PESC,
and thereby also explain its main theoretical ideas. The presentation is given in
Haskell code. Some of it may be readable to anyone familiar with type theory or some
functional programming language; the Haskell home page mentioned in references
gives more information.


\subsection{The structure of the source code}

Here is a wc dump of the \PESC\ source files. Each file
contains a module of the same name. The sizes of the modules may
still change a bit, but the total code will not greatly differ from
the indicated 1396 lines, which includes comments and empty lines. 
\begin{verbatim}
    178     923    5967 Axioms.hs      -- operations on nonlogical axioms
    193     966    7249 Calculi.hs     -- predefined calculi
    198     968    7582 Editor.hs      -- dialogue and command language
    103     598    4410 Interaction.hs -- refinement, proof search, etc
    169     993    6906 Natural.hs     -- natural deduction
     82     429    2613 PSequent.hs    -- parsing
    112     616    3756 PrSequent.hs   -- printing 
    161     939    4249 PrelSequent.hs -- prelude : auxiliary functions
    200     980    6457 Sequent.hs     -- basic data types and functions
   1396    7412   49189 total
\end{verbatim}


\subsection{The module {\tt Sequent}---basic data types and functions}

This module defines the {\bf abstract syntax}\ of sequents, formulae, 
singular terms, proofs, etc. On the level of abstract syntax, all these
expressions are Haskell data objects, which could also be called
{\bf syntax trees}. In user input and \PESC\ output, syntax trees are represented by
{\bf strings}\ of characters. The relation between syntax trees and strings
is defined by the {\bf parsing}\ and {\bf printing}\ functions explained in 
the next section. 

With the exception of the communication with the user, \PESC\ always operates 
on syntax trees and not on strings. There are lots of operations defined by
pattern matching on the basic types of syntax trees: {\tt Sequent}, {\tt Formula},
{\tt Term}, {\tt Proof}, {\tt AbsRule}. %{\tt }
The operations defined in the module {\tt Sequent}\ include substitution and
tree-node numbering. Most of this material is unsurprising. But there are a
couple of features that identify \PESC\ as an editor precisely for
shared-context multiset calculi.

The definition that expresses the top-down determinacy of \PESC\ (cf.\ Section 1)
is the definition of abstract inference rules:
\begin{verbatim}
  type AbsRule = Sequent -> Maybe [Either Sequent Ident]
\end{verbatim}
A rule is a function that takes a conclusion sequent as its argument and returns
either a list of premises or a failure. Returning an empty list of premises means
that the proof of the conclusion is complete. The premises can be either sequents 
or parametre identifiers. (We already argued, in Section 2.2 above, that parametres
are really premises.)

Sequents are treated as pairs of multisets of formulae. The definition of the 
type of sequents is as pairs of lists rather than multisets: the multiset aspect
is implicit in various functions that consider variants of these lists obtained by
making one formula in turn the active formula. In most cases, this is enough, and
it is not necessary to consider all permutations.

Some calculi restrict the succedents of sequents to be single formulae. \PESC\ does
not use a distinct type for these calculi: it simply is the property of certain 
calculi, such as G3ip, that the rules never require nor produce multi-succedent
sequents.



\subsection{The modules {\tt PSequent}\ and {\tt PrSequent}---parsing and printing}

Parsing follows the top-down method explained in Wadler (1985). The basic 
{\bf parsing combinators}\ are defined in the module {\tt PrelSequent}.
We have chosen to define parsing independently of an environment that could, for
instance, give lists of constants and predicates: because of this, we have had
to distinguish between different types of identifiers, as explained in 
Section 4.3 above. In some crucial places, such as in parsing formulae with infix
connectives, the parser is optimized by left factorization and thus does not 
strictly follow the abstract syntax.

The printing functions first produce terminal output for \PESC\ sessions.
A simple string transformer {\tt makeLatex}\ produces \LaTeX\ code from this,
except in one case: proof trees. The proof trees printed on terminal do not
always represent well the structure of the proof, and they become practically
unreadable when the proofs grow. \LaTeX\ proof trees are printed separately, 
directly from syntax trees.


\subsection{The module {\tt Calculi}---predefined calculi}

Some intuitionistic and classical calculi are defined directly as
sets of abstract rules. Because abstract rules are functions, they cannot
be read from separate files at runtime, but must be compiled into \PESC. 
We also considered a special syntax that could be used for reading calculi
from files, but finally restricted this to nonlogical axioms: there is,
after all, so much variation and irregularity in the rules, that the
syntax and the operations on it become complicated, and are not
guaranteed to cope with ``arbitrary sequent-calculus rules''.

At least when using \PESC\ in the interpreter hugs, it should be easy to 
experiment with new calculi by editing the file {\tt Calculi.hs}. No other
files need be edited.


\subsection{The module {\tt Interaction}---refinement and proof search}

This module defines the central proof search functions: refinement,
instantiation, undoing, and automatic search. All of these operations
are based on the underlying operation of replacing a subtree by another tree,
\begin{verbatim}
  replace :: Proof -> [Int] -> Proof -> Proof
\end{verbatim}
where lists of integers denote subtrees. Another underlying operation is to
list all those rules of a clculus that apply to a given sequent,
\begin{verbatim}
  applicableRules :: 
    AbsCalculus -> Sequent -> [((Ident,AbsRule),Obligations)]
\end{verbatim}
Automatic proving calls this function in performing top-down proof search,
following the methods of Wadler (1985) just like parsing.


\subsection{The module {\tt Editor}---dialogue and command language}

This is the {\tt Main}\ module, which also contains the definition of the
command language and the parser of the latter, as well as the help message.
The central function is the execution of commands, 
\begin{verbatim}
  exec :: Command -> (AbsCalculus,Proof) -> (Proof,String)
\end{verbatim}
which works in an environment of a current calculus and a current proof,
and possibly changes it, by calling the functions of {\tt Interaction}.


\subsection{The module {\tt Axioms}---operations on nonlogical axioms}

This module defines a format for a special kind of rules, nonlogical axioms.
These axioms are just pairs of lists of atoms, which are easy read from
familiar logical notation (cf.\ Section 4.4 above). Their translation into
abstract \PESC\ rules is somewhat complicated because of the presence
of implicit parametres.


\subsection{The module {\tt Natural}---translation into natural deduction}

This module defines a translation from sequent calculus proofs
to natural deduction in a rather {\em ad hoc} way, which only applies
to the calculi G3i and G3ip. Systematizing and extending the
translation would be a welcome contribution to \PESC!


\section{Known bugs and defects}

\PESC\ is an experimental system still under development.
Contributions to this Section are thus welcome. As of \today, the most
urgent problems are the following:
\bequ
The printing of proof trees on terminal is inadequate for large branching
proofs.

In \LaTeX\ printing, it would be nice to put multicharacter identifiers
into \verb6\mbox6es.

The equality rules in {\tt Calculus}\ are unfinished.
\enqu
In the Haskell code, those lines that we find obscure or suboptimal have been
suffixed by three hyphens, \verb6---6.


\section*{References}

\setlength{\parindent}{0mm}
\setlength{\parskip}{1.2mm}

G.\ Gentzen.
Untersuchungen \"uber das logische Schlie\ss en.
{\em Mathematische Zeitschrift}, 39:176--210 and 405--431, 1934.

HBC home page, \verb6http://www.cs.chalmers.se/~augustss/hbc/hbc.html6

Hugs home page, \verb6http://www.haskell.org/hugs/6

S.\ Kleene,
{\em Introductin to Metamathematics},
North-Holland, Amsterdam, 1952.

L.\ Magnusson,
{\em The implementation of ALF---a proof editor based on Martin-L\"of's
mono\-morphic type theory with explicit substitution},
PhD thesis, Department of Computing Science, 
Chalmers University of Technology,
Gothenburg, 1994.

P.\ {Martin-L\"{o}f}.
{\em Intuitionistic Type Theory}.
Bibliopolis, Naples, 1984.

S.\ Negri \&\ J.\ von Plato,
``Structural Proof Theory'', manuscript, University of Helsinki, 1999.

\PESC\ home page, to appear.

K.\ Sch\"utte, 
{\em Proof Theory}, Springer, 1977.

M.\ Tatsuta,
{\tt proof.sty}\ (Proof Figure Macros), \LaTeX\ style file, 1990.

A.\ Troelstra \&\ H.\ Schwichtenberg,
{\em Basic Proof Theory},
Cambridge University Press,
1996.

P.\ Wadler,
``How to replace failure by a list of successes'',
{\em Proceedings of Conference on Functional Programming
Languages and Computer Architecture}, pp.\ 113--128, 
{\em Lecture Notes in Computer Science}\ vol.\ 201.
Springer, Heidelberg, 1985.


\newpage




\section*{Appendix: Some calculi}

The rules are from Negri \&\ von Plato (1999).

\newcommand{\him}{\supset}
\newcommand{\seq}{\Rightarrow}
\newcommand{\himmih}{\supset \subset}
\newcommand{\dotseq}{\stackrel{\stackrel{\vdots}{ \ }}{\Rightarrow}}


\newcommand{\noin}{\noindent}
\newcommand{\vsk}{\vskip 5pt}
\newcommand{\npar}{\vskip 5pt \noindent}
\newcommand{\tbf}{\textbf}
\newcommand{\Gm}{\Gamma}
\newcommand{\Dlt}{\Delta}
\newcommand{\eps}{\mbox{\scriptsize $\ \in \ $}}
\newcommand{\nleqs}{\mbox{\scriptsize $\ \nleqslant\ $}}
\newcommand{\less}{\mbox{\scriptsize $\ < \ $}}
\newcommand{\neqs}{\mbox{\scriptsize $\ \neq \ $}}
\newcommand{\eqs}{\mbox{\scriptsize $\ = \ $}}

\centerline{\mbox{{\bf G$n$ip}}}
\[
\begin{array}{lll}
\qquad
P,\Gamma\seq P  
\qquad\qquad
&
\infer[\scriptstyle L\bot]{\bot,\Gamma\seq C}{}
\qquad\qquad
&
\infer[\scriptstyle R\him]{\Gamma\seq A\him B}{A,\Gamma\seq B } 
\end{array}
\]
\[
\begin{array}{ll}
\infer[\scriptstyle L\&]{A\& B,\Gamma\seq C}{A,B,\Gamma\seq C}
\qquad\qquad
&\infer[\scriptstyle R\&]{\Gamma\seq A\& B}{\Gamma\seq A & \Gamma\seq B}\\*[10pt]
\infer[\scriptstyle L\vee]{A\vee B,\Gamma\seq C}{A,\Gamma\seq C & B,\Gamma\seq C}
\qquad
&\infer[\scriptstyle R\vee1]{\Gamma\seq A\vee B}{\Gamma\seq A} \qquad
\infer[\scriptstyle R\vee2]{\Gamma\seq A\vee B}{\Gamma\seq B}
\end{array}
\]

%\vskip 10pt

\centerline{\mbox{{\bf G3ip = G$n$ip +}}}
\[
\infer[\scriptstyle L\him]{A\him B,\Gamma\seq C}{A\him B,\Gamma\seq A & B,\Gamma\seq C}
\]

\centerline{\mbox{{\bf G4ip = G$n$ip + }}}
\[
\begin{array}{ll}
\qquad \infer[\scriptstyle L0\him]{P,P\him B,\Gm\seq E}{P,B,\Gm\seq E}
\qquad
&
\infer[\scriptstyle L\&\him]{C\& D\him B,\Gm\seq E}{C\him(D\him B),\Gm\seq E}
\\*[10pt]
\qquad \infer[\scriptstyle L\lor\him]{C\lor D\him B,\Gm\seq E}{C\him B, D\him B,\Gm\seq E}  
\qquad
&
\infer[\scriptstyle L\him\him]{(C\him D)\him B,\Gm\seq E}{C,D\him B,\Gm\seq D
& B,\Gm\seq E}
\end{array}
\]

\centerline{\mbox{{\bf G3cp}}}
\[
\begin{array}{ll}
%%\mbox{{\bf Logical axiom:}}
%%\\*[10pt]
P,\Gamma\seq \Dlt,P 
\qquad\qquad &
\infer[\scriptstyle L\bot]{\perp,\Gamma\seq \Dlt}{}
\\*[10pt]
%%\mbox{{\bf Logical rules:}}
%%\\*[10pt]
\infer[\scriptstyle L\&]{A\& B,\Gamma\seq \Dlt}{A,B,\Gamma\seq \Dlt}
\qquad\qquad
&\infer[\scriptstyle R\&]{\Gamma\seq \Dlt,A\& B}{\Gamma\seq \Dlt,A & \Gamma\seq\Dlt, B}\\*[10pt]
\infer[\scriptstyle L\vee]{A\vee B,\Gamma\seq \Dlt}{A,\Gamma\seq \Dlt & B,\Gamma\seq \Dlt}
\qquad
&\infer[\scriptstyle R\vee]{\Gamma\seq \Dlt,A\vee B}{\Gamma\seq \Dlt,A,B} 
\\*[10pt]
\infer[\scriptstyle L\him]{A\him B,\Gamma\seq \Dlt}{\Gamma\seq \Dlt,A & B,\Gamma\seq \Dlt}
\qquad
&\infer[\scriptstyle R\him]{\Gamma\seq \Dlt,A\him B}{A,\Gamma\seq \Dlt,B } 
\end{array}
\]

\centerline{\mbox{{\bf G3i = G3ip +}}}
\[
\begin{array}{llll}
\infer[\scriptstyle L\forall]{\forall xA, \Gm \seq C}
                             {A(t/x),\forall xA, \Gm \seq C}
& \infer[\scriptstyle R\forall]{\Gamma\seq \forall xA}{\Gamma\seq A(y/x)} 
&
\infer[\scriptstyle L\exists]{\exists xA,\Gamma\seq C}{A(y/x),\Gamma\seq C}

& \infer[\scriptstyle R\exists]{\Gamma\seq \exists xA}{\Gamma\seq A(t/x)}
\end{array}
\]

\centerline{\mbox{{\bf G3c = G3cp +}}}
\[
\begin{array}{ll}
\infer[\scriptstyle L\forall]{\forall xA, \Gm \seq \Dlt}
                             {A(t/x),\forall xA,\Gm \seq \Dlt}
\qquad\qquad \qquad
& \infer[\scriptstyle R\forall]{\Gamma\seq \Dlt , \forall xA}{\Gamma\seq \Dlt,A(y/x)} 
\\*[10pt]
\infer[\scriptstyle L\exists]{\exists xA,\Gamma\seq \Dlt}{A(y/x),\Gamma\seq \Dlt}
\qquad \qquad \qquad
& \infer[\scriptstyle R\exists]{\Gamma\seq \Dlt,\exists xA}{\Gamma\seq \Dlt,\exists xA,A(t/x)}
\end{array}
\]


\end{document}
